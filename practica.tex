\documentclass[a4paper, twoside]{article}
\usepackage[utf8]{inputenc} % Especifica la codificación de caracteres de los documentos.
\usepackage[spanish]{babel} % Indica que el documento se escribirá en español.
\usepackage[top=3cm, bottom=2.5cm, inner=1.5cm, outer=2.5cm]{geometry} % Márgenes personalizados
\usepackage{subfiles} % Paquete para incluir el preambulo en los sub archivos.
\usepackage{afterpage} % Permite añadir páginas despues de una página dada.
\usepackage{hyperref} % Permite incluir enlaces en los archivos.
\usepackage{lastpage} % Paquete para poder contabilizar el total de páginas del documento.
\usepackage{fancyhdr} % Permite personalizar los header y footer del documento.
\usepackage{tikz} % Permite incluir los diagramas exportados con DIA
% Defino la ruta de los paquetes personalizados para el apunte

% Matemáticas
\usepackage{amsmath} % para escribir matrices
\usepackage{amsfonts} % \mathbb por ej.
\usepackage{amssymb} % \triangleq por ej. http://ia.wikipedia.org/wiki/Wikipedia:LaTeX_symbols


\newcommand{\rutapaquetes}{./paquetes-apunte}

\usepackage{\rutapaquetes/caratula} % Caratula personalizada (cargada desde caratula.sty)
\usepackage[mostrarrevisores]{\rutapaquetes/colaboradores} % Seccion de colaboradores (cargada y creada con colaboradores.sty)
\usepackage{\rutapaquetes/historial} % Seccion de historial de cambios (cargada y creada con historial.sty)

% Define los estilos de los enlaces interpretados por el paquete hyperref
\hypersetup{
    colorlinks=true,   % false: boxed links; true: colored links
    linkcolor=black,   % color of internal links (change box color with linkbordercolor)
    citecolor=green,   % color of links to bibliography
    filecolor=magenta, % color of file links
    urlcolor=blue,     % color of external links
}

% Define los directorios de las imágenes y gráficos
\graphicspath{ {./} {\rutapaquetes/} }

% Define el pagestyle personalizado
\pagestyle{fancy}
\fancyhf{}
\renewcommand{\sectionmark}[1]{\markboth{}{\thesection\ \ #1}}
% Define header para pagina par
\fancyhead[ER]{\rightmark}
% Define header para pagina impar
\fancyhead[OL]{\rightmark}
% Define footer para pagina par
\fancyfoot[EL]{Ejercicios - Sincronismo} % Nombre del apunte a la izquierda
\fancyfoot[ER]{Página \thepage\ de \pageref{LastPage}} % Numero de pagina a la derecha
% Define footer para pagina impar
\fancyfoot[OL]{Página \thepage\ de \pageref{LastPage}} % Numero de pagina a la izquierda
\fancyfoot[OR]{Ejercicios - Sincronismo} % Nombre del apunte a la derecha

\renewcommand{\footrulewidth}{0.4pt} % Agrego linea que separa el footer

\newcommand{\nombremateria}{66.78 - Comunicaciones Digitales y Analógicas} % Defino el comando "\nombremateria" para no harcodear el nombre en varios lugares.

% Configura la caratula
\materia{\nombremateria}
\tipoapunte{Apunte Práctico}
\tema{Sincronismo}

\begin{document}
% Página en blanco agregada después de la carátula
%\afterpage{
%	\null
%	\thispagestyle{empty}%
%	\addtocounter{page}{-1}%
%	\newpage}
\maketitle % Genera la carátula

\tableofcontents % Genera el índice

\subfile{\rutapaquetes/sobre-el-proyecto.tex} % Inlcuye informacion sobre el proyecto FIUBA Apuntes

% Insertar aquí el contenido del apunte. A continuacion hay secciones a modo de ejemplo.
\paragraph{Aclaración} En todos los desarrollos estamos asumiendo que vale el modelo linealizado del PLL. Si no tenés idea de que significa esto, leete las notas de Cioffi\cite{Cio6}.

\section{Ejercicio 1 - PLL de primer orden}
Considere un sistema de recuperación de portadora que utiliza un PLL. Asuma que el error de fase es lo suficientemente pequeño como para que el modelo lineal del PLL dado por la Figura 1 sea aceptable. Asuma que el filtro de lazo L(s) tiene una transferencia de orden cero L(s) = K.

\subsection{Item 1}\paragraph{Enunciado}Determine la transferencia del sistema a lazo cerrado: \[G(s) = \frac{\hat{\Theta}(s)}{\Theta(s)}\] y la transferencia del error: \[G_e(s)=\frac{E(s)}{\Theta(s)}\] determinando los valores de K para los cuales el lazo es estable.

\begin{figure}[h]
	\centering
	% Graphic for TeX using PGF
% Creator: Dia v0.97.2
% CreationDate: Thu Jan 22 22:36:39 2015
% For: Fernando Danko
% \usepackage{tikz}
% The following commands are not supported in PSTricks at present
% We define them conditionally, so when they are implemented,
% this pgf file will use them.
%
% El código producido por DIA fue modificado a mano para mejorar el aspecto.
% Cambios:
% * Las etiquetas se pasaron a LaTeX. Se aumentó el tamaño con \huge{$ec$}
% y con \LARGE{$ec$}.
% * Se movieron las posiciones de las etiquetas

\ifx\du\undefined
  \newlength{\du}
\fi
\setlength{\du}{15\unitlength}
\begin{tikzpicture}
\pgftransformxscale{1.000000}
\pgftransformyscale{-1.000000}
\definecolor{dialinecolor}{rgb}{0.000000, 0.000000, 0.000000}
\pgfsetstrokecolor{dialinecolor}
\definecolor{dialinecolor}{rgb}{1.000000, 1.000000, 1.000000}
\pgfsetfillcolor{dialinecolor}
\definecolor{dialinecolor}{rgb}{1.000000, 1.000000, 1.000000}
\pgfsetfillcolor{dialinecolor}
\fill (12.500000\du,6.000000\du)--(12.500000\du,8.000000\du)--(16.000000\du,8.000000\du)--(16.000000\du,6.000000\du)--cycle;
\pgfsetlinewidth{0.100000\du}
\pgfsetdash{}{0pt}
\pgfsetdash{}{0pt}
\pgfsetmiterjoin
\definecolor{dialinecolor}{rgb}{0.000000, 0.000000, 0.000000}
\pgfsetstrokecolor{dialinecolor}
\draw (12.500000\du,6.000000\du)--(12.500000\du,8.000000\du)--(16.000000\du,8.000000\du)--(16.000000\du,6.000000\du)--cycle;
% setfont left to latex
\definecolor{dialinecolor}{rgb}{0.000000, 0.000000, 0.000000}
\pgfsetstrokecolor{dialinecolor}
\node at (14.250000\du,7.180000\du){\LARGE{$L(s)$}};
\definecolor{dialinecolor}{rgb}{1.000000, 1.000000, 1.000000}
\pgfsetfillcolor{dialinecolor}
\fill (13.500000\du,9.500000\du)--(13.500000\du,11.500000\du)--(14.920000\du,11.500000\du)--(14.920000\du,9.500000\du)--cycle;
\pgfsetlinewidth{0.100000\du}
\pgfsetdash{}{0pt}
\pgfsetdash{}{0pt}
\pgfsetmiterjoin
\definecolor{dialinecolor}{rgb}{0.000000, 0.000000, 0.000000}
\pgfsetstrokecolor{dialinecolor}
\draw (13.500000\du,9.500000\du)--(13.500000\du,11.500000\du)--(14.920000\du,11.500000\du)--(14.920000\du,9.500000\du)--cycle;
% setfont left to latex
\definecolor{dialinecolor}{rgb}{0.000000, 0.000000, 0.000000}
\pgfsetstrokecolor{dialinecolor}
\node at (14.210000\du,10.680000\du){\LARGE{$\frac{1}{s}$}};
\pgfsetlinewidth{0.100000\du}
\pgfsetdash{}{0pt}
\pgfsetdash{}{0pt}
\pgfsetmiterjoin
\pgfsetbuttcap
{
\definecolor{dialinecolor}{rgb}{0.000000, 0.000000, 0.000000}
\pgfsetfillcolor{dialinecolor}
% was here!!!
\pgfsetarrowsend{stealth}
{\pgfsetcornersarced{\pgfpoint{0.000000\du}{0.000000\du}}\definecolor{dialinecolor}{rgb}{0.000000, 0.000000, 0.000000}
\pgfsetstrokecolor{dialinecolor}
\draw (16.000000\du,7.000000\du)--(17.050000\du,7.000000\du)--(17.050000\du,10.500000\du)--(14.920000\du,10.500000\du);
}}
% setfont left to latex
\definecolor{dialinecolor}{rgb}{0.000000, 0.000000, 0.000000}
\pgfsetstrokecolor{dialinecolor}
\node[anchor=west] at (2.000000\du,7.000000\du){\huge{$\Theta(t)$}};
% setfont left to latex
\definecolor{dialinecolor}{rgb}{0.000000, 0.000000, 0.000000}
\pgfsetstrokecolor{dialinecolor}
\node[anchor=west] at (2.000000\du,10.500000\du){\huge{$\hat{\Theta}(t)$}};
\pgfsetlinewidth{0.100000\du}
\pgfsetdash{}{0pt}
\pgfsetdash{}{0pt}
\pgfsetbuttcap
{
\definecolor{dialinecolor}{rgb}{0.000000, 0.000000, 0.000000}
\pgfsetfillcolor{dialinecolor}
% was here!!!
\pgfsetarrowsend{stealth}
\definecolor{dialinecolor}{rgb}{0.000000, 0.000000, 0.000000}
\pgfsetstrokecolor{dialinecolor}
\draw (10.000000\du,7.000000\du)--(12.500000\du,7.000000\du);
}
% setfont left to latex
\definecolor{dialinecolor}{rgb}{0.000000, 0.000000, 0.000000}
\pgfsetstrokecolor{dialinecolor}
\node[anchor=west] at (14.210000\du,10.500000\du){};
\pgfsetlinewidth{0.100000\du}
\pgfsetdash{}{0pt}
\pgfsetdash{}{0pt}
\pgfsetbuttcap
{
\definecolor{dialinecolor}{rgb}{0.000000, 0.000000, 0.000000}
\pgfsetfillcolor{dialinecolor}
% was here!!!
\pgfsetarrowsend{stealth}
\definecolor{dialinecolor}{rgb}{0.000000, 0.000000, 0.000000}
\pgfsetstrokecolor{dialinecolor}
\draw (13.500000\du,10.500000\du)--(5.000000\du,10.500000\du);
}
\pgfsetlinewidth{0.100000\du}
\pgfsetdash{}{0pt}
\pgfsetdash{}{0pt}
\pgfsetbuttcap
{
\definecolor{dialinecolor}{rgb}{0.000000, 0.000000, 0.000000}
\pgfsetfillcolor{dialinecolor}
% was here!!!
\pgfsetarrowsend{stealth}
\definecolor{dialinecolor}{rgb}{0.000000, 0.000000, 0.000000}
\pgfsetstrokecolor{dialinecolor}
\draw (9.000000\du,10.500000\du)--(9.000000\du,8.000000\du);
}
\pgfsetlinewidth{0.100000\du}
\pgfsetdash{}{0pt}
\pgfsetdash{}{0pt}
\pgfsetbuttcap
{
\definecolor{dialinecolor}{rgb}{0.000000, 0.000000, 0.000000}
\pgfsetfillcolor{dialinecolor}
% was here!!!
\pgfsetarrowsend{stealth}
\definecolor{dialinecolor}{rgb}{0.000000, 0.000000, 0.000000}
\pgfsetstrokecolor{dialinecolor}
\draw (5.000000\du,7.000000\du)--(8.000000\du,7.000000\du);
}
% setfont left to latex
\definecolor{dialinecolor}{rgb}{0.000000, 0.000000, 0.000000}
\pgfsetstrokecolor{dialinecolor}
\node[anchor=west] at (3.500000\du,10.000000\du){};
% setfont left to latex
\definecolor{dialinecolor}{rgb}{0.000000, 0.000000, 0.000000}
\pgfsetstrokecolor{dialinecolor}
\node[anchor=west] at (6.500000\du,6.200000\du){\huge{$+$}};
% setfont left to latex
\definecolor{dialinecolor}{rgb}{0.000000, 0.000000, 0.000000}
\pgfsetstrokecolor{dialinecolor}
\node[anchor=west] at (7.500000\du,8.500000\du){\huge{$-$}};
% setfont left to latex
\definecolor{dialinecolor}{rgb}{0.000000, 0.000000, 0.000000}
\pgfsetstrokecolor{dialinecolor}
\node[anchor=west] at (10.000000\du,6.200000\du){\huge{e(t)}};
\pgfsetlinewidth{0.100000\du}
\pgfsetdash{}{0pt}
\pgfsetdash{}{0pt}
\pgfsetbuttcap
\pgfsetmiterjoin
\pgfsetlinewidth{0.100000\du}
\pgfsetbuttcap
\pgfsetmiterjoin
\pgfsetdash{}{0pt}
\definecolor{dialinecolor}{rgb}{1.000000, 1.000000, 1.000000}
\pgfsetfillcolor{dialinecolor}
\pgfpathellipse{\pgfpoint{9.000000\du}{7.000000\du}}{\pgfpoint{1.000000\du}{0\du}}{\pgfpoint{0\du}{1.000000\du}}
\pgfusepath{fill}
\definecolor{dialinecolor}{rgb}{0.000000, 0.000000, 0.000000}
\pgfsetstrokecolor{dialinecolor}
\pgfpathellipse{\pgfpoint{9.000000\du}{7.000000\du}}{\pgfpoint{1.000000\du}{0\du}}{\pgfpoint{0\du}{1.000000\du}}
\pgfusepath{stroke}
\pgfsetbuttcap
\pgfsetmiterjoin
\pgfsetdash{}{0pt}
\definecolor{dialinecolor}{rgb}{0.000000, 0.000000, 0.000000}
\pgfsetstrokecolor{dialinecolor}
\draw (8.292900\du,6.292900\du)--(9.707100\du,7.707100\du);
\pgfsetbuttcap
\pgfsetmiterjoin
\pgfsetdash{}{0pt}
\definecolor{dialinecolor}{rgb}{0.000000, 0.000000, 0.000000}
\pgfsetstrokecolor{dialinecolor}
\draw (8.292900\du,7.707100\du)--(9.707100\du,6.292900\du);
\end{tikzpicture}

	\caption{Diagrama en bloques de un PLL linealizado.}
	\label{fig:pll_lineal}
\end{figure}

Siguiendo el esquema de la Figura \ref{fig:pll_lineal}, tenemos que
\begin{align}
\hat{\Theta}(s) &= \frac{1}{s}\, L(s)\, E(s)\nonumber\\
				&= \frac{K}{s} (\Theta(s) - \hat{\Theta}(s))\nonumber\\ 
G(s) = \frac{\hat{\Theta}(s)}{\Theta(s)} &= \frac{s}{s+K} \label{ec:ej1:G}
\end{align}

y luego:

\begin{align}
G_e(s) 	&= \frac{\Theta(s)-\hat{\Theta}(s)}{\Theta(s)}\nonumber\\
		&= \frac{s}{K+s} \label{ec:ej1:Ge}
\end{align}

Analizando la ecuación \ref{ec:ej1:G}, se observa que la transferencia del PLL posee un polo en $s=-K$. Para que el sistema sea estable todos los polos deben estar en el semiplano izquierdo, por ende debe cumplirse que $K \ge 0$.

\subsection{Item 2} \paragraph{Enunciado}Realice un diagrama de Bode de la magnitud de $G(s)$ y $G_e(s)$ indicando los valores más representativos. ¿Qué tipo de transferencia es cada una de ellas?

A esta altura de la vida, se ve claro que la transferencia del sistema $G(s)$ es un pasa-bajos, mientras que la transferencia del error $G_e(s)$ es un pasa-altos, ambos de primer orden.

\subsection{Item 3}  \paragraph{Enunciado}Suponga que la portadora local tiene un offset de fase y de frecuencia constantes respecto de la portadora recibida, es decir, que $\theta(t) = (\theta_0 + \Delta\omega\, t)\, u(t)$. Determine el valor de $e(t)$ en estado estacionario.

Dado que $G_e(s)$ es estable, podemos hablar de error estacionario. Transformando $\theta(t)$ obtenemos:

\begin{align}
	\Theta(s) = \theta_0 \cdot \frac{1}{s} + \Delta\omega \cdot \frac{1}{s^2}
\end{align}

Y según el modelo linealizado (Figura \ref{fig:pll_lineal}):
\begin{align}
	E(s) 	&= \Theta(s)\, G(s) \nonumber\\
			&= \theta_0 \frac{1}{K+s} + \Delta\omega \frac{1}{s(K+s)}
\end{align}

Y empleando el Teorema del Valor Final (dado que el sistema es estable):
\begin{align}
	e(+\infty) 	&= \lim\limits_{s \to 0} s E(s) \nonumber\\
				&= \lim\limits_{s \to 0} \left( \theta_0 \frac{s}{K+s} + \Delta\omega \frac{1}{K+s} \right) \nonumber\\
				&= \frac{\Delta\omega}{K}
\end{align}

\subsection{Item Extra} \paragraph{Enunciado} Ya que estamos, vamos a calcular el \textit{Rango de Enganche} del PLL.

La condición de enganche es, básicamente:

\begin{align}
	\Delta\omega \le \pi L(0) \label{ec:rango_enganche}
\end{align}

Que en este caso se reduce a: $\Delta\omega \le K\pi$.

\section{Ejercicio 5 - Offset en PLL}
Se utiliza un PLL para hallar la fase de una señal senoidal cercana a $1MHz$. El error en
la frecuencia puede ser de $0.01\%$. Se utiliza un PLL de primer orden con filtro de lazo $L(s)$
para este trabajo. Se admite un error en estado estacionario que no debe superar el $5\%$.

\subsection{Item 1}
\paragraph{Enunciado} Hallar el valor de $K$ que verifique que el error de fase es menor a $5\%$, o lo que es lo
mismo menor a $0.01\pi$.

El error de frecuencia es $\Delta f = 0.01\%\, 1MHz = 100Hz$, o lo que es lo mismo, $\Delta \omega = 2\pi\, \Delta f = 200\pi\, 1/s$.

En primer lugar, hay que cumplir la condición de enganche. Dado que $L(s) = K$ tenemos:
\begin{align*}
\Delta \omega &\le \pi L(0)\\
200 \pi &\le \pi K\\
K &> 200
\end{align*}

Con esa ganancia el sistema permanece enganchado en estas condiciones, pero falta cumplir la condición de error en estado estacionario, esto es:

\begin{align*}
e(+\infty) &< 0.01 \pi\\
\lim\limits_{s\to 0} sE(s) = \frac{\Delta \omega}{K} &< 0.01 \pi\\
K > 2\times 10^4
\end{align*}

Vemos que con $K > 2\times 10^4$ se cumplen ambas condiciones.

\subsection{Item 2}
\paragraph{Enunciado} Suponga ahora que la frecuencia es cercana a $10GHz$ en lugar de $1MHz$. El PLL resultante, ¿es estable?

Usando el mismo PLL: $L(s) = 2\times 10^4$, ahora con $f_c = 10GHz$ tenemos que $\Delta \omega = 2\times 10^5 \pi\, 1/s$. Pero al verificar la condición de enganche, se debe cumplir:

\begin{align*}
\Delta \omega &\le \pi L(0)\\
\Delta \omega &\le 2\times 10^4 \pi\, 1/s
\end{align*}

Por ende, el PLL no engancha. (Aunque es estable, en el sentido BIBO como sistema LTI, ya que tiene los polos en el semiplano izquierdo, pero no funcionaría como PLL).

\subsection{Item 3}
 Si el PLL es inestable, ¿podría resolverse el problema utilizando un PLL de segundo orden? ¿Cómo resolvería el problema sin recurrir a un PLL de orden mayor?
 
 Para usar un PLL de primer orden, simplemente basta con diseñarlo de nuevo y obtener un valor de $K$ enorme. Esto puede ser un problema a la hora de implementarlo.
 
 Se podría diseñar un PLL de segundo orden que sea más ``potable'' tecnológicamente que el de primer orden, ya que $G(s)$ pasaría de ser un pasabajos a un pasabandas; por ende al tener un menor ancho de banda sería menos propenso a los problemas con el ruido.

% Bibliografía utilizada en el apunte
\newpage
\newcommand{\bibliographyname}{Bibliografía} % Defino el nombre de la sección de la bibliografía
\addcontentsline{toc}{section}{\bibliographyname} % Agrego la bibliografía en el índice
\renewcommand\refname{\bibliographyname} % Renombro a la bibliografía (por default es 'Referencias')
\begin{thebibliography}{X}
	\bibitem{Cio6} \textsc{John Cioffi}, \textit{ Digital Communication: Signal Processing}, 2013. Course Reader. Chapter 6.  \href{http://www.stanford.edu/group/cioffi/ee379a/}{http://www.stanford.edu/group/cioffi/ee379a/}

\end{thebibliography}


% Incluir los nombres de las personas que han colaborado en la creación del apunte
\colaborador{Fernando Iván Danko (fdanko@fi.uba.ar)}
%\colaborador{Colaborador 2}
\revisor{}{}
\makeseccioncolaboradores % Crea la seccion de colaboradres

% Incluir el historial de cambios
\revision{22/01/2015}{Versión inicial.}

\end{document}